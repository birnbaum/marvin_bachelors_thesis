This chapter presents the hardware testbed. The assembled setup system is
capable of dynamically adjusting its computational load relative to the on-site
energy production by photo-voltaic (PV) modules. Section
\ref{sec:hardware_components} lists all main physical components and
demonstrates the conformity with the requirements from section
\ref{sec:testbed_requirements}. Section \ref{sec:hardware_assembly} shows the
assembly of all components and explains their interrelations. Subsequently
section \ref{sec:software_setup} shows the software setup, necessary to ensure
functionality of the components, data transmission and operability. Section
\ref{sec:implementation_of_dvfs} presents the implementation of DVFS for this
testbed.

\section{Hardware Components}
\label{sec:hardware_components}

The testbed is mainly composed of the following hardware. As the compute node,
the single-board computer \emph{Raspberry Pi 3b+} serves as a viable choice due
to its low energy consumption, cost effectiveness and a wide range of hardware
applications \footnote{\url{https://www.raspberrypi.com}}. Renewable energy is
produced by 4 PV modules, each outputting 330 mA at 6 V. Excess energy produced
by the PV modules is stored in a 3.7 V, 6600 mAh lithium-ion polymer (LiPo)
battery as backup energy source in case of suboptimal solar conditions.
Ultimately a component, connecting all components listed above, is needed to
measure the energy production of the PV modules, the battery's state of charge
and the energy consumption of the compute node. \emph{SwitchDoc Labs} developed
\emph{SunControl}, an inexpensive solar power controller board, among multiple
other things capable of these tasks \cite{switchdoc_suncontrol} and is therefore
ideal for this testbed.

In an idle state, with the \textsc{HDMI} driver and \textsc{LED}s disabled to
preserve energy, the node consumes about 400 mA (2 W). With the same
configuration under load, the node consumes up to 980 mA (3.7 W), yielding a
quotient between power consumption under load and idle of 2.45.
% TODO provide figure and proof

\section{Hardware Assembly}
\label{sec:hardware_assembly}

\section{Software Setup}
\label{sec:software_setup}

%For dynamic resource management, the frequency and voltage of the Raspberry Pi's
%\textsc{ARM CPU} can be scaled dynamically e.g. with the \texttt{cpufreq}
%subsystem\footnote{\url{https://community.arm.com/oss-platforms/w/docs/528/cpufreq-dvfs}}.

\section{Implementation of DVFS}
\label{sec:implementation_of_dvfs}

