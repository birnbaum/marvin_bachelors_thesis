This chapter elaborates on relevant concepts revolving around energy-aware
resource management in edge and fog computing. Section
\ref{sec:edge_and_fog_computing_paradigms} introduces the motivation and
necessity for edge and fog computing paradigms. Section
\ref{sec:testbeds_in_edge_and_fog_computing_environments} focuses on the
development process of edge and fog infrastructure and outlines the essential
use of testbeds. Section \ref{sec:dynamic_resource_management} presents common
techniques in dynamic resource management whilst focusing on energy proportional
computing. Since the renewable energy source in the hardware testbed is not able
to supply uninterrupted power, section
\ref{sec:energy-aware_resource_management} outlines specific techniques used in
energy-aware resource management. Dynamic voltage and frequency scaling will be
the primary technique used in the hardware testbed and is therefore elaborated
on further in section \ref{sec:dynamic_voltage_and_frequency_scaling}.

\section{Edge and Fog Computing Paradigms}
\label{sec:edge_and_fog_computing_paradigms}

In 2012 Xia et al. describes the Internet of Things as the networked
interconnection of everyday objects, which are often equipped with ubiquitous
intelligence \cite{xia2012}. Since then, IoT devices have spread to all aspects
of our everyday lives. The rise of IoT resulted in a significant increase of
devices connected to the internet and will, with its current rate of
development, soon grow to a point, our current network structure cannot manage.
In specific, major challenges that are posed by traditional device to cloud
connections include the diminishing availability of bandwidth and high latency
due to physical distance.

Fog and edge computing paradigms pose a viable solution to the problem stated
above. They induce a highly virtualized platform that provides compute, storage,
and networking services between end devices and traditional Cloud Computing Data
Centers, typically, but not exclusively located at the edge of network
\cite{bonomi2012}. With the physical proximity of provided services, the load on
cloud servers are relieved, throughput is subsequently raised and latency for
the end device highly lowered.

\section{Testbeds in Edge and Fog Computing Environments}
\label{sec:testbeds_in_edge_and_fog_computing_environments}

\section{Dynamic Resource Management}
\label{sec:dynamic_resource_management}

\subsection{Energy-Aware Resource Management}
\label{sec:energy-aware_resource_management}

\subsection{Dynamic Voltage and Frequency Scaling}
\label{sec:dynamic_voltage_and_frequency_scaling}

