Edge computing is a promising, emerging paradigm in the area of distributed
systems and while still primarily being a theoretical concept, the rise of new
domains like Internet of Things (IoT) establishes numerous areas of application
\cite{premsankar2018}. Because of the decentralized nature of edge computing,
new devices at the edge of the network are essential for many of these
approaches. In some cases though, edge devices do not have access to the
electrical grid and require on-site energy generation. Some examples include
portable weather stations, smart watering and metering in autonomous farming or
numerous sensors in smart city designs. Large-scale data centers already profit
from integration of on-site renewable energy generation and are able to achieve
significant cost reductions while also reducing their greenhouse gas emissions
\cite{tu2013}. In its current state however, renewable energy generation is
rather volatile and unable to supply sufficient uninterrupted power on its own
\cite{rostirolla2022}, resulting in a problem for edge devices without a
connection to the electrical grid. If devices under these circumstances aim to
operate self-sufficiently and maximize uptime, their resources need to be
managed dynamically, relative to current on-site renewable energy production.

To allow for dynamic resource management, matching the system's energy
consumption to the on-site production, numerous approaches for power management
pose a viable option. Power management, in this case energy proportional
computing, can generally be pursued on a hardware- and software-level
\cite{barroso2007}. On a hardware-level, dynamic voltage and frequency scaling
(DVFS), the dynamic adjustment of both voltage and frequency to reduce dynamic
power consumption, is vital for this intention as the \textsc{CPU} traditionally
consumes the most power in a system.

Recent research in the area of energy-aware resource management in edge/fog
computing utilizing DVFS for power management mainly relies on simulations to
predict a real-world outcome \cite{wiesner2022, karimiafshar2020, toor2019}.
Most simulators make the assumption, that computational load can be adjusted in
a way that energy consumption perfectly matches the energy production
\cite{leaf, cloudsim, ifogsim, chameleon}. While the reasoning for using
simulations as opposed to real hardware testbeds may be justified and opportune
for most research projects, it remains unclear how close these assumptions are
to reality and how this may change the accuracy of the predictions and
consequently research outcomes.

\section{Testbed Requirements}
\label{sec:testbed_requirements}

This bachelor's thesis proposes a hardware testbed for renewable-aware resource
management in edge computing, capable of dynamically adjusting its computational
load relative to the on-site renewable energy production. In order to examine
the assumption of energy-aware simulators, that computational load is adjustable
so that energy consumption is matched to production, this thesis proposes to
compare simulations derived from real-world data to this physical hardware
testbed. To provide the necessary data to compare the simulations to, a testbed
with the following properties is required.

\begin{enumerate}
    \item Renewable energy is produced and consumed respectively.
    \item Excess energy produced, can be stored and retrieved if the production
        fails to provide sufficient power.
    \item \[ \frac{P_{\text{load}}}{P_{\text{idle}}} \geq 2 .\]
        Adjusting the consumption to current production is more meaningful, in
        regard to energy savings, if the quotient between the power consumption
        under load and idle is as high as possible. In order to work properly
        with the data provided, the lower bound 2 was chosen.
    \item The current energy production and resources' power draw can be
        measured in Volts and Amperes respectively. The amount of stored power
        can be measured.
    \item Resources can be managed dynamically. This is necessary for adjusting
        the consumption depending on the current production of renewable energy.
\end{enumerate}

\section{Thesis Outline}

This thesis is divided into 6 chapters.

Chapter \ref{chap:background} introduces the fundamental concepts discussed in
subsequential chapters. First, edge and fog computing paradigms and their
necessity for future information and communications technology (ICT) are
presented. Second, the respective use cases of physical and simulation testbeds
in edge and fog computing environments are outlined. Third, dynamic resource
management, and specifically energy proportional computing for our use case,
techniques are presented and DVFS, as the main technique for our physical
testbed, along with its implementation in the Linux kernel, will be outlined.

Chapter \ref{chap:related_work} first reviews related research in the area of
energy-aware resource management utilizing DVFS. The utilization of simulations
in all presented papers leads to the described problem which is subsequently
outlined. Then, these simulations are presented and aforementioned problem is
demonstrated.

Chapter \ref{chap:testbed} presents the hardware testbed. First, all hardware
components used and their conformity with the requirements listed in section
\ref{sec:testbed_requirements} are presented. Second, the assembly of the
individual components and their relations to each other are described. Third,
the setup of the software, necessary to ensure functionality of the components,
data transmission and operability, is presented. Fourth, the implementation of
DVFS, customized for this testbed, is outlined.

Chapter \ref{chap:evaluation} evaluates the testbed by assessing the accuracy of
its capability to adjusting its computational load relative to the on-site
energy generation. An error analysis is conducted and reviewed in regards to
collected data by the testbed. The subsequent consequences for conducted
research in the area of energy-aware resource management are discussed.

Chapter \ref{chap:conclusion} concludes the thesis by summarizing the main
points and contributions.

