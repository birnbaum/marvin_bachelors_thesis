Edge computing is a promising, emerging paradigm in the area of distributed
systems and while still primarily being a theoretical concept, the rise of new
domains like Internet of Things (IoT) establishes numerous areas of application
\cite{premsankar2018}. Because of the decentralized nature of edge computing,
new devices at the edge of the network are essential for many of these
approaches. In some cases though, edge devices do not have access to the
electrical grid and require on-site energy generation. %TODO give example
Large-scale data centers already profit from integration of on-site renewable
energy generation and are able to achieve significant cost reductions while also
reducing their greenhouse gas emissions. \cite{tu2013}. In its current state
however, renewable energy generation is rather volatile and unable to supply
sufficient uninterrupted power on its own \cite{rostirolla2022}, resulting in a
problem for edge devices without a connection to the electrical grid. If devices
under these circumstances aim to operate self-sufficiently and maximize uptime,
their resources need to be managed dynamically, relative to current on-site
renewable energy production.

To allow for dynamic resource management, matching the system's energy
consumption to the on-site production, numerous approaches for power management
pose a viable option. Power management, in this case energy proportional
computing, can generally be pursued on a hardware- and software-level
\cite{barroso2007}. On a hardware-level, dynamic voltage and frequency scaling
(DVFS), the dynamic adjustment of both voltage and frequency to reduce dynamic
power consumption, is vital for this intention as the \textsc{CPU} traditionally
consumes the most power in a system.

Recent Research in the area of energy-aware resource management in edge/fog
computing utilizing DVFS for power management mainly relies on simulations to
predict a real-world outcome \cite{wiesner2022, karimiafshar2020, toor2019}.
Most simulators make the assumption that computational load can be adjusted in a
way that energy consumption perfectly matches the energy production \cite{leaf,
cloudsim, ifogsim, chameleon}. While the reasoning for using simulations as
opposed to real hardware testbeds may be justified and opportune for most
research projects, it remains unclear how close these assumptions are to reality
and how this may change the accuracy of the predictions and consequently
research outcomes.

\section{Testbed Requirements}

This bachelor's thesis proposes a hardware testbed for renewable-aware resource
management in edge computing, capable of dynamically adjusting its computational
load relative to the on-site renewable energy production. In order to examine
the assumption of energy-aware simulators, that computational load is adjustable
so that energy consumption is matched to production, this thesis proposes to
compare simulations derived from real-world data to this physical hardware
testbed. To provide the necessary data to compare the simulations to, a testbed
with the following properties is required.

\begin{enumerate}
    \item Renewable energy is produced and consumed respectively.
    \item Excess energy produced, can be stored and retrieved if the production
        fails to provide sufficient power.
    \item \[ \frac{P_{\text{load}}}{P_{\text{idle}}} \geq 2 .\]
        Adjusting the consumption to current production is more meaningful, in
        regard to energy savings, if the quotient between the power consumption
        under load and idle is as high as possible. In order to work properly
        with the data provided, the lower bound 2 was chosen.
    \item The current energy production and resources' power draw can be
        measured in Volts and Amperes respectively. The amount of stored power
        can be measured.
    \item Resources can be managed dynamically. This is necessary for adjusting
        the consumption depending on the current production of renewable energy.
\end{enumerate}

\section{Thesis Outline}
